%\input{beamer-template1} % More colorful presentation
\input{beamer-template2} % Traditional colors presentation
\begin{document}

\author[Kostyszyn]{Dejan Kostyszyn}
\date{Date of presentation}
\title{Title of presentation}
\subtitle{Subtitle of the presentation}
\institute{Albert-Ludwigs-Universität Freiburg}


\begin{frame}[plain, noframenumbering]
	\titlepage
\end{frame}


\begin{frame}[plain, noframenumbering]{Table of contents}
\tableofcontents[sectionstyle=show/show]
\end{frame}


\section[Theorems]{A list of Theorems}
\begin{frame}
	This is a presentation especially designed for the Albert-Ludwigs-Universität Freiburg im Breisgau. To do so, I used the beamer-template \cite{ctan_beamer}.
\end{frame}


\subsection[Def., Cor., Thm., Proof]{Definition, Corollary, Theorem and Proof}
\begin{frame}{Theorems you can use (Part 1)}
	\begin{definition}[A definition]
		This is a definition.
	\end{definition}
	\begin{corollary}[A very important corollary]
		This is a \alert{very important} corollary.
	\end{corollary}
    \begin{theorem}[A theorem]
    	This is a theorem.
    \end{theorem}
    \begin{proof}[Proof for the theorem]
    	This is a proof.
    \end{proof}
\end{frame}


\begin{frame}{Theorems you can use (Part 2)}
	\begin{proposition}[A proposition]
		This is a proposition.
	\end{proposition}
	\begin{lemma}[A lemma]
		This is a lemma.
	\end{lemma}
	\begin{lemma}[Another lemma]
		This is another lemma.
	\end{lemma}
\end{frame}


\subsection[Include images]{Including a lot of images}
\begin{frame}[plain]
	\begin{figure}
		\centering
		\includegraphics[scale=.2]{img/Logo1000px.png}
	\end{figure}
\end{frame}


\begin{frame}{Some text and an image}
\begin{columns}[T]
    \begin{column}{.5\textwidth}
     \begin{block}{A textblock}
		Lorem ipsum dolor sit amet, consetetur sadipscing elitr, sed diam nonumy eirmod tempor invidunt ut labore et dolore magna aliquyam erat, sed diam voluptua. At vero eos et accusam et justo duo dolores et ea rebum. Stet clita kasd gubergren, no sea takimata sanctus est Lorem ipsum dolor sit amet. Lorem ipsum dolor sit amet, consetetur sadipscing elitr, \dots
    \end{block}
    \end{column}
    \begin{column}{.5\textwidth}
    \begin{block}{An image}
    \includegraphics[width=\textwidth]{img/Logo1000px.png}
    \end{block}
    \end{column}
  \end{columns}
\end{frame}


\section{Text and Mathematics}
\begin{frame}{Text}
	\begin{itemize}
		\item Just
		\item some
		\item items
		\item and
		\item now
		\item maths \dots
	\end{itemize}
\end{frame}


\subsection[Maths]{Mathematics}
\begin{frame}{Mathematics (Part 1)}
 	\begin{corollary}[Kleiner Gauß]
 	\[
 	\sum_{k=1}^n k = \frac{n(n + 1)}{2}
 	\]
 	\end{corollary}
\end{frame}


\begin{frame}{Mathematics (Part 1)}
 	\begin{proof}
	 	\begin{align}
 			n = 1: & \sum_{k=1}^1 k & 					&= \frac{1 \cdot (1 + 1)}{2} = 1\\
 			n: & \sum_{k=1}^n k &  					&= \frac{n \cdot (n + 1)}{2}, \quad n \in \mathbb{N}\\
 			n \rightarrow n+1: & \sum_{k=1}^{n+1} k &   &= \frac{n(n + 1)}{2} + (n + 1)\\
 			& & 										&= \frac{(n + 1) \cdot ((n + 1) + 1)}{2}
 		\end{align}
 	\end{proof}
\end{frame}


\begin{frame}{Thanks for your attention!}
	\printbibliography
\end{frame}
\end{document}
